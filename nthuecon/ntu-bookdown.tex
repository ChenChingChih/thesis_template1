%fix pandoc 2.8 update


  \documentclass[oneside]{book}
%\documentclass[]{book}

\usepackage{lmodern}
\usepackage{setspace}
\setstretch{1.5}
\usepackage{amssymb,amsmath}
\usepackage{ifxetex,ifluatex}
\usepackage{fixltx2e} % provides \textsubscript
\ifnum 0\ifxetex 1\fi\ifluatex 1\fi=0 % if pdftex
  \usepackage[T1]{fontenc}
  \usepackage[utf8]{inputenc}
\else % if luatex or xelatex
  \ifxetex
    \usepackage{xltxtra,xunicode}
  \else
    \usepackage{fontspec}
  \fi
  \defaultfontfeatures{Ligatures=TeX,Scale=MatchLowercase}


\fi
% use upquote if available, for straight quotes in verbatim environments
\IfFileExists{upquote.sty}{\usepackage{upquote}}{}
% use microtype if available
\IfFileExists{microtype.sty}{%
\usepackage{microtype}
\UseMicrotypeSet[protrusion]{basicmath} % disable protrusion for tt fonts
}{}
\usepackage[a4paper, left=1.18in, right=1.18in, top=1.18in, bottom=0.787in]{geometry}
\usepackage[unicode=true]{hyperref}
\hypersetup{
            pdftitle={臺大論文模板},
            pdfauthor={廖永賦},
            pdfborder={0 0 0},
            breaklinks=true}
\urlstyle{same}  % don't use monospace font for urls
\usepackage{color}
\usepackage{fancyvrb}
\newcommand{\VerbBar}{|}
\newcommand{\VERB}{\Verb[commandchars=\\\{\}]}
\DefineVerbatimEnvironment{Highlighting}{Verbatim}{commandchars=\\\{\}}
% Add ',fontsize=\small' for more characters per line
\usepackage{framed}
\definecolor{shadecolor}{RGB}{248,248,248}
\newenvironment{Shaded}{\begin{snugshade}}{\end{snugshade}}
\newcommand{\AlertTok}[1]{\textcolor[rgb]{0.94,0.16,0.16}{#1}}
\newcommand{\AnnotationTok}[1]{\textcolor[rgb]{0.56,0.35,0.01}{\textbf{\textit{#1}}}}
\newcommand{\AttributeTok}[1]{\textcolor[rgb]{0.77,0.63,0.00}{#1}}
\newcommand{\BaseNTok}[1]{\textcolor[rgb]{0.00,0.00,0.81}{#1}}
\newcommand{\BuiltInTok}[1]{#1}
\newcommand{\CharTok}[1]{\textcolor[rgb]{0.31,0.60,0.02}{#1}}
\newcommand{\CommentTok}[1]{\textcolor[rgb]{0.56,0.35,0.01}{\textit{#1}}}
\newcommand{\CommentVarTok}[1]{\textcolor[rgb]{0.56,0.35,0.01}{\textbf{\textit{#1}}}}
\newcommand{\ConstantTok}[1]{\textcolor[rgb]{0.00,0.00,0.00}{#1}}
\newcommand{\ControlFlowTok}[1]{\textcolor[rgb]{0.13,0.29,0.53}{\textbf{#1}}}
\newcommand{\DataTypeTok}[1]{\textcolor[rgb]{0.13,0.29,0.53}{#1}}
\newcommand{\DecValTok}[1]{\textcolor[rgb]{0.00,0.00,0.81}{#1}}
\newcommand{\DocumentationTok}[1]{\textcolor[rgb]{0.56,0.35,0.01}{\textbf{\textit{#1}}}}
\newcommand{\ErrorTok}[1]{\textcolor[rgb]{0.64,0.00,0.00}{\textbf{#1}}}
\newcommand{\ExtensionTok}[1]{#1}
\newcommand{\FloatTok}[1]{\textcolor[rgb]{0.00,0.00,0.81}{#1}}
\newcommand{\FunctionTok}[1]{\textcolor[rgb]{0.00,0.00,0.00}{#1}}
\newcommand{\ImportTok}[1]{#1}
\newcommand{\InformationTok}[1]{\textcolor[rgb]{0.56,0.35,0.01}{\textbf{\textit{#1}}}}
\newcommand{\KeywordTok}[1]{\textcolor[rgb]{0.13,0.29,0.53}{\textbf{#1}}}
\newcommand{\NormalTok}[1]{#1}
\newcommand{\OperatorTok}[1]{\textcolor[rgb]{0.81,0.36,0.00}{\textbf{#1}}}
\newcommand{\OtherTok}[1]{\textcolor[rgb]{0.56,0.35,0.01}{#1}}
\newcommand{\PreprocessorTok}[1]{\textcolor[rgb]{0.56,0.35,0.01}{\textit{#1}}}
\newcommand{\RegionMarkerTok}[1]{#1}
\newcommand{\SpecialCharTok}[1]{\textcolor[rgb]{0.00,0.00,0.00}{#1}}
\newcommand{\SpecialStringTok}[1]{\textcolor[rgb]{0.31,0.60,0.02}{#1}}
\newcommand{\StringTok}[1]{\textcolor[rgb]{0.31,0.60,0.02}{#1}}
\newcommand{\VariableTok}[1]{\textcolor[rgb]{0.00,0.00,0.00}{#1}}
\newcommand{\VerbatimStringTok}[1]{\textcolor[rgb]{0.31,0.60,0.02}{#1}}
\newcommand{\WarningTok}[1]{\textcolor[rgb]{0.56,0.35,0.01}{\textbf{\textit{#1}}}}
\usepackage{longtable,booktabs}
% Fix footnotes in tables (requires footnote package)
\IfFileExists{footnote.sty}{\usepackage{footnote}\makesavenoteenv{long table}}{}
\let\oldhref=\href
% Make links footnotes instead of hotlinks:
\renewcommand{\href}[2]{#2\footnote{\url{#1}}}
\IfFileExists{parskip.sty}{%
\usepackage{parskip}
}{% else
\setlength{\parindent}{0pt}
\setlength{\parskip}{6pt plus 2pt minus 1pt}
}
\setlength{\emergencystretch}{3em}  % prevent overfull lines
\providecommand{\tightlist}{%
  \setlength{\itemsep}{0pt}\setlength{\parskip}{0pt}}
\setcounter{secnumdepth}{5}

% set default figure placement to htbp
\makeatletter
\def\fps@figure{htbp}
\makeatother

\usepackage{pdfpages}
\usepackage{titlesec}
\usepackage{titletoc}
\usepackage{booktabs}
\usepackage{apptools}
\usepackage{float}
\usepackage[section]{placeins}

\usepackage[heading, fontset = none]{ctex}
\ctexset{appendix/name={\appendixname\space}}

\usepackage[fontsize=12pt]{scrextend}

% 如果想將每頁的頁碼置於中間下方,uncomment 下兩行
%\usepackage{fancyhdr}
%\pagestyle{fancy}

% Eng font-family
\setmainfont[
  Path=latex/,
  BoldFont={TimesNewRomanBold},
  ItalicFont={TimesNewRomanItalic},
  BoldItalicFont={TimesNewRomanBoldItalic}
]{TimesNewRoman}

% Trad Ch font-family
\setCJKmainfont[Path=latex/,AutoFakeBold=2.5,AutoFakeSlant=.3]{kaiti}
\setCJKmonofont[Path=latex/]{NotoSansMonoCJKtc}

% Special font: IPA
\newfontfamily{\ipa}[Path=latex/,AutoFakeBold=2.5,AutoFakeSlant=.3]{IPAfont} % Font for IPA symbols
\DeclareTextFontCommand{\ipatext}{\ipa}


%中文自動換行
\XeTeXlinebreaklocale "zh"
%文字的彈性間距
\XeTeXlinebreakskip = 0pt plus 1pt



\renewcommand{\figurename}{圖}
\renewcommand{\tablename}{表}
\renewcommand{\contentsname}{目錄}
\renewcommand{\listfigurename}{圖目錄}
\renewcommand{\listtablename}{表目錄}
\renewcommand{\appendixname}{附錄}
%\renewcommand{\bibname}{參考資料}

% deal with nuts floating figures
\renewcommand{\textfraction}{0.05}
\renewcommand{\topfraction}{0.8}
\renewcommand{\bottomfraction}{0.8}
\renewcommand{\floatpagefraction}{0.75}

\title{臺大論文模板}
\author{廖永賦}
\date{一月 11, 2021}


\usepackage{fontspec}
%使用xeCJK,其他的還有CJK或是xCJK
\usepackage{xeCJK}
\usepackage{bm}

% Set the default fonts
% See https://tug.org/pipermail/xetex/2011-March/020226.html for fontspec
% % \setmainfont[
%   Path=latex/,
%   BoldFont={TimesNewRomanBold.ttf},
%   ItalicFont={TimesNewRomanItalic.ttf},
%   BoldItalicFont={TimesNewRomanBoldItalic.ttf}
% ]{TimesNewRoman.ttf}
% 
% %     \setCJKmainfont[Path=latex/,AutoFakeBold=2.5,AutoFakeSlant=.3]{kaiti}
%     \setCJKmonofont[Path=latex/]{NotoSansMonoCJKtc}
% 


% IPA support (Works with linguisticsdown)
% 

\usepackage{xcolor}
\usepackage{transparent}

\usepackage{tikz}
\usepackage[printwatermark]{xwatermark}
\newsavebox\mybox
\savebox\mybox{\tikz[]\node[opacity=0.2]{\includegraphics{watermark.png}};}
\newwatermark*[
  allpages,
  %angle=45,
  scale=0.18,
  xpos=6.3725cm,
  ypos=10.8225cm
]{\usebox\mybox}






\begin{document}


\includepdf[pages={1}, scale=1]{front_matter/front_matter.pdf}

\clearpage
\pagenumbering{roman}

\phantomsection
\addcontentsline{toc}{chapter}{口試委員會審定書}
\includepdf[pages={1}, scale=1]{certification-scan.pdf}


\phantomsection
\chapter*{誌謝}
非常感謝網路上各個默默耕耘開發 open source 專案的大大們。
非常感謝網路上各個默默耕耘開發 open source 專案的大大們。
非常感謝網路上各個默默耕耘開發 open source 專案的大大們。

沒有這些既存的資源,這份模板是不可能出現的。沒有這些既存的資源,這份模板是不可能出現的。沒有這些既存的資源,這份模板是不可能出現的。沒有這些既存的資源,這份模板是不可能出現的。
\addcontentsline{toc}{chapter}{誌謝}


\phantomsection
\chapter*{摘要}
摘要\textbf{第 5 行}開始而且不能是空行。摘要\textbf{第 5 行}開始而且不能是空行。摘要\textbf{第 5 行}開始而且不能是空行。摘要\textbf{第 5 行}開始而且不能是空行。

新段落要在前面空一行。新段落要在前面空一行。新段落要在前面空一行。新段落要在前面空一行。新段落要在前面空一行。
\bigbreak

\noindent
\textbf{關鍵字:} 第二行開始、R Markdown、Bookdown、可重製研究
\addcontentsline{toc}{chapter}{中文摘要}

\phantomsection
\chapter*{Abstract}
The first line of the abstract starts on \textbf{line 5} and must not be blank. The first line of the abstract starts on \textbf{line 5} and must not be blank. The first line of the abstract starts on \textbf{line 5} and must not be blank.

A new paragraph of the abstract. A new paragraph of the abstract. A new paragraph of the abstract. A new paragraph of the abstract. A new paragraph of the abstract. A new paragraph of the abstract.
\bigbreak

\noindent
\textbf{Keywords:} Line 2, R Markdown, Bookdown, Reproducible Research
\addcontentsline{toc}{chapter}{英文摘要}


{
\setcounter{tocdepth}{1}
\tableofcontents
%\phantomsection
%\addcontentsline{toc}{chapter}{\contentsname}
}

\newpage

\listoftables
\phantomsection
\addcontentsline{toc}{chapter}{\listtablename}
\newpage

\listoffigures
\phantomsection
\addcontentsline{toc}{chapter}{\listfigurename}
\newpage

% Set independent linestretch for code chunks
\let\oldShaded=\Shaded
\let\endoldShaded=\endShaded
\renewenvironment{Shaded}{
      \begin{spacing}{1}\begin{oldShaded}
    }
  {
  \end{oldShaded}
  \end{spacing}
  }

\clearpage
\pagenumbering{arabic}

\hypertarget{install}{%
\chapter{安裝}\label{install}}

\begin{Shaded}
\begin{Highlighting}[]
\NormalTok{devtools}\OperatorTok{::}\KeywordTok{install_github}\NormalTok{(}\StringTok{"liao961120/ntuthesis"}\NormalTok{)}
\end{Highlighting}
\end{Shaded}

\hypertarget{latex}{%
\section{LaTeX}\label{latex}}

若已有管理、安裝 LaTeX 套件經驗者,可忽略。

若電腦尚未安裝 LaTeX,可安裝 R 的 tinytex 套件:

\begin{Shaded}
\begin{Highlighting}[]
\KeywordTok{install.packages}\NormalTok{(}\StringTok{'tinytex'}\NormalTok{)}
\NormalTok{tinytex}\OperatorTok{::}\KeywordTok{install_tinytex}\NormalTok{()}
\end{Highlighting}
\end{Shaded}

在輸出 R Markdown 時,tinytex 會自動安裝缺少的 LaTeX 套件。因此,第一次輸出 PDF 可能會需要一些時間。

\hypertarget{export-thesis}{%
\chapter{輸出論文}\label{export-thesis}}

\hypertarget{import-template}{%
\section{匯入論文模板}\label{import-template}}

開啟 RStudio,選取左上方 \texttt{File} \textgreater{} \texttt{New\ File} \textgreater{} \texttt{R\ Markdown}:

或是直接在 console 執行:

\begin{Shaded}
\begin{Highlighting}[]
\NormalTok{rmarkdown}\OperatorTok{::}\KeywordTok{draft}\NormalTok{(}\StringTok{"project_name"}\NormalTok{,}
                 \DataTypeTok{template =} \StringTok{"ntu_bookdown"}\NormalTok{,}
                 \DataTypeTok{package =} \StringTok{"ntuthesis"}\NormalTok{)}
\end{Highlighting}
\end{Shaded}

接著需要將該資料夾變更為 bookdown 專案。這可以用 RStudio 左上方 \texttt{File} \textgreater{} \texttt{New\ Project} \textgreater{} \texttt{Existing\ Directory} 達成,或直接使用下方指令(working dir 需是專案資料夾):

\begin{Shaded}
\begin{Highlighting}[]
\NormalTok{ntuthesis}\OperatorTok{::}\KeywordTok{init_proj}\NormalTok{()  }\CommentTok{# init working dir as proj.}
\end{Highlighting}
\end{Shaded}

詳細的檔案結構,見 \ref{dir-structure}。

\hypertarget{edit-front-matter}{%
\section{編輯封面}\label{edit-front-matter}}

在\texttt{\_person-info.yml}輸入個人資料後,執行:

\begin{Shaded}
\begin{Highlighting}[]
\NormalTok{ntuthesis}\OperatorTok{::}\KeywordTok{comp_front}\NormalTok{()}
\end{Highlighting}
\end{Shaded}

即會在 \texttt{front-matter/} 生成封面所需的檔案。以使用者的角度而言,除了 \texttt{front-matter/certification.pdf} 以外,\texttt{front-matter/} 中的其它檔案不須理會。\texttt{certification.pdf} 是空白(未簽名)的「口試委員審定書」。

已簽名的「口試委員審定書」,將檔案命名為 \texttt{certification-scan.pdf} 並放在專案資料夾的最頂層。

\hypertarget{compile-thesis}{%
\section{Compile 論文}\label{compile-thesis}}

接著在 console 執行下方指令:

\begin{Shaded}
\begin{Highlighting}[]
\NormalTok{bookdown}\OperatorTok{::}\KeywordTok{render_book}\NormalTok{(}\StringTok{"index.Rmd"}\NormalTok{, }\StringTok{"bookdown::gitbook"}\NormalTok{)}
\NormalTok{bookdown}\OperatorTok{::}\KeywordTok{render_book}\NormalTok{(}\StringTok{"index.Rmd"}\NormalTok{, }\StringTok{"bookdown::bookdown::pdf_book"}\NormalTok{)}
\end{Highlighting}
\end{Shaded}

如此便會在 \texttt{\_book/} 中生成完整的論文(gitbook 和 PDF 格式)。

\hypertarget{write-thesis}{%
\chapter{論文撰寫}\label{write-thesis}}

\hypertarget{dir-structure}{%
\section{檔案結構}\label{dir-structure}}

執行以下指令後(詳見 \ref{import-template})

\begin{Shaded}
\begin{Highlighting}[]
\NormalTok{rmarkdown}\OperatorTok{::}\KeywordTok{draft}\NormalTok{(}\StringTok{"project_name"}\NormalTok{,}
                 \DataTypeTok{template =} \StringTok{"ntu_bookdown"}\NormalTok{,}
                 \DataTypeTok{package =} \StringTok{"ntuthesis"}\NormalTok{)}
\end{Highlighting}
\end{Shaded}

即會匯入論文模板,以下是論文模板的檔案結構(已簡化)。

\begin{Shaded}
\begin{Highlighting}[]
\AttributeTok{├── project_name.Rmd}\CommentTok{     # Useless, please delete it}
\CharTok{|}
\AttributeTok{├── R/}\CommentTok{                   # code chunk root dir, put R scripts and data here}
\AttributeTok{├── figs/}\CommentTok{                # Put figures to include in the thesis here}
\CharTok{|}
\AttributeTok{├── index.Rmd}\CommentTok{            # Book Layout (font, watermark, biblio, ...)}
\AttributeTok{├── _acknowledge.Rmd}\CommentTok{     # acknowledgement}
\AttributeTok{├── _abstract-en.Rmd}\CommentTok{     # abstract}
\AttributeTok{├── _abstract-zh.Rmd}\CommentTok{     # Same as above, but in Chinese}
\CharTok{|}
\AttributeTok{├── 01-intro.Rmd}\CommentTok{         # Chapter 1 content}
\AttributeTok{├── 02-literature.Rmd}\CommentTok{    # Chapter 2 content}
\AttributeTok{├── 03-method.Rmd}\CommentTok{        # Chapter 3 content}
\AttributeTok{├── 99-references.Rmd}\CommentTok{    # Don't need to edit}
\AttributeTok{├── ref.bib}\CommentTok{              # References}
\AttributeTok{├── cite-style.csl}\CommentTok{       # Citation style}
\CharTok{|}
\AttributeTok{├── _bookdown.yml}\CommentTok{        # label names in gitbook; Rmd files order}
\AttributeTok{├── _output.yml}\CommentTok{          # preamble, pandoc args, cite-pkg}
\CharTok{|}
\AttributeTok{├── watermark.pdf}\CommentTok{        # 臺大浮水印 (PDF 右上角)}
\AttributeTok{├── _person-info.yml}\CommentTok{      # Info to generate front matter}
\AttributeTok{├── certification-scan.pdf}\CommentTok{  # 已簽名'口試委員審查書'}
\AttributeTok{└── front_matter}
\AttributeTok{    └── certification.pdf}\CommentTok{   # 空白'口試委員審查書'}
\end{Highlighting}
\end{Shaded}

\hypertarget{index-rmd}{%
\section{\texorpdfstring{\texttt{index.Rmd}}{index.Rmd}}\label{index-rmd}}

\texttt{index.Rmd} 是設定論文內文格式的地方,包含 yaml 以及 R setup code chunk。此模板將 code chunk 預設的 working directory 改成 \texttt{R/}\footnote{預設是 Rmd 檔所在的位置。},如此較符合一般寫 Rscript 的邏輯\footnote{例如,使用相對路徑匯入資料時,一般會以 Rscript 所在的位置作為基準。}。若要更改此設定,至 setup code chunk 更改 \texttt{knitr::opts\_knit\$set(root.dir=\textquotesingle{}R\textquotesingle{})}。

\hypertarget{write-lang}{%
\section{撰寫語言}\label{write-lang}}

若使用\textbf{英文}撰寫論文,需修改 \texttt{\_output.yml}、\texttt{\_bookdown.yml} 這兩個檔案的內容。

\hypertarget{output.yml}{%
\subsection{\texorpdfstring{\texttt{\_output.yml}}{\_output.yml}}\label{output.yml}}

將 \texttt{in\_header:\ latex/preamble-zh.tex} 改為 \texttt{in\_header:\ latex/preamble-en.tex}:

\begin{Shaded}
\begin{Highlighting}[]
\AttributeTok{bookdown:}\FunctionTok{:pdf_book}\KeywordTok{:}
\AttributeTok{  }\FunctionTok{includes}\KeywordTok{:}
\AttributeTok{    }\FunctionTok{in_header}\KeywordTok{:}\AttributeTok{ latex/preamble-en.tex}
\end{Highlighting}
\end{Shaded}

\hypertarget{bookdown.yml}{%
\subsection{\texorpdfstring{\texttt{\_bookdown.yml}}{\_bookdown.yml}}\label{bookdown.yml}}

\texttt{\_bookdown.yml} 中,可以對標籤的名稱進行定義。這裡的設定與 PDF 輸出無關,只與 gitbook 輸出格式有關。因此,若無需使用 gitbook 輸出,可忽略此段。

此外,\texttt{\_bookdown.yml} 亦可設定 Rmd 檔在輸出文件中的順序。若無設定,就會依序檔名排序\footnote{此模板即未進行設定,因此第一章的內容寫在 \texttt{01-xxx.Rmd} 就會自動排在第一。而若檔名以底線開頭(\texttt{\_})則會被忽略。更多內容詳見 \href{https://bookdown.org/yihui/bookdown/usage.html}{bookdown}。}。

在以下設定中,可使 gitbook 輸出的章節(順序)與 PDF 不同。

\begin{Shaded}
\begin{Highlighting}[]
\FunctionTok{rmd_files}\KeywordTok{:}
\AttributeTok{  }\FunctionTok{html}\KeywordTok{:}\AttributeTok{ }\KeywordTok{[}\StringTok{"index.Rmd"}\KeywordTok{,}\AttributeTok{ }\StringTok{"abstract.Rmd"}\KeywordTok{,}\AttributeTok{ }\StringTok{"intro.Rmd"}\KeywordTok{]}
\AttributeTok{  }\FunctionTok{latex}\KeywordTok{:}\AttributeTok{ }\KeywordTok{[}\StringTok{"abstract.Rmd"}\KeywordTok{,}\AttributeTok{ }\StringTok{"intro.Rmd"}\KeywordTok{]}
\end{Highlighting}
\end{Shaded}

\hypertarget{bib-cite}{%
\section{文獻引用}\label{bib-cite}}

R Markdown 在文章中插入引用文獻的功能承繼 Pandoc。完整的使用見 \href{https://rmarkdown.rstudio.com/authoring_bibliographies_and_citations.html}{R Markdown 官方說明} 。

此模板目前產生文獻格式的方法是依靠 \href{https://github.com/jgm/pandoc-citeproc}{Pandoc citeproc},因此,文獻格式是依據 \texttt{cite-style.csl}\footnote{此模板提供的 \texttt{cite-style.csl} 是 APA 英文第六版。此外,\url{http://blog.pulipuli.info/2011/05/zoteroapa.html} 亦有提供 APA 中文版的引用格式。需注意的是 Pandoc \textbf{不支援雙語 csl} (\url{http://blog.pulipuli.info/2014/08/zoteroapa-zotero-citation-style-apa.html})。} 產生的。使用者可至 \href{https://www.zotero.org/styles}{Zotero Style Repository} 下載所需的 csl 檔並覆蓋專案資料夾中的 \texttt{cite-style.csl}。

\hypertarget{ref-bib}{%
\subsection{\texorpdfstring{\texttt{ref.bib}}{ref.bib}}\label{ref-bib}}

\texttt{.bib} 檔的產生方式可以由 Endnote, Zotero, JabRef 等書目管理軟體匯出。匯出後,將檔名命名為 \texttt{ref.bib} 放在專案資料夾\footnote{或是可以自訂檔名,並到 \texttt{index.Rmd} yaml 中的 \texttt{bibliography:\ ref.bib} 更改 \texttt{ref.bib} 檔名。此外,亦可使用多個 \texttt{.bib} 檔:\texttt{bibliography:\ {[}ref1.bib,\ ref2.bib,\ ref3.bib{]}}。}。

\texttt{.bib} 內的一篇引用資料會類似:

\begin{Shaded}
\begin{Highlighting}[]
\VariableTok{@article}\NormalTok{\{}\OtherTok{leung2008}\NormalTok{,}
  \DataTypeTok{title}\NormalTok{ = \{Multicultural Experience Enhances Creativity: \{\{The\}\} When and How.\},}
  \DataTypeTok{volume}\NormalTok{ = \{63\},}
  \DataTypeTok{issn}\NormalTok{ = \{1935-990X(Electronic),0003-066X(Print)\},}
  \DataTypeTok{doi}\NormalTok{ = \{10.1037/0003-066X.63.3.169\},}
  \DataTypeTok{number}\NormalTok{ = \{3\},}
  \DataTypeTok{journaltitle}\NormalTok{ = \{American Psychologist\},}
  \DataTypeTok{date}\NormalTok{ = \{2008\},}
  \DataTypeTok{pages}\NormalTok{ = \{169-181\},}
  \DataTypeTok{keywords}\NormalTok{ = \{*Cognition,*Creativity,*Culture (Anthropological),*Experiences (Events),Multiculturalism\},}
  \DataTypeTok{author}\NormalTok{ = \{Leung, Angela Ka-yee and Maddux, William W. and Galinsky, Adam D. and Chiu, Chi-yue\}}
\NormalTok{\}}
\end{Highlighting}
\end{Shaded}

其中第一行的 \texttt{leung2008} 即為 citation key。透過 \texttt{@citekey}(\texttt{@leung2008})即可在文獻中插入 citation。匯出論文時,文末會自動產生引用的文獻。

\hypertarget{cite-syntax}{%
\subsection{引用語法}\label{cite-syntax}}

\href{https://github.com/crsh/citr}{\texttt{citr}} 是一個方便使用者插入引用文獻的 R 套件,讓使用者能透過 GUI 插入文獻。

當需要更複雜的引用格式,如標示第幾頁,可以修改透過 \texttt{citr} 插入的語法:

\begin{itemize}
\tightlist
\item
  \texttt{Some\ text\ {[}@citekey{]}.}

  \begin{itemize}
  \tightlist
  \item
    Some text (Leung, Maddux, Galinsky, \& Chiu, \protect\hyperlink{ref-leung2008}{2008}).
  \end{itemize}
\item
  \texttt{@citekey\ Some\ text}

  \begin{itemize}
  \tightlist
  \item
    Leung et al. (\protect\hyperlink{ref-leung2008}{2008}) Some text
  \end{itemize}
\item
  \texttt{@citekey\ {[}p.\ 20{]}\ Some\ text.}

  \begin{itemize}
  \tightlist
  \item
    Leung et al. (\protect\hyperlink{ref-leung2008}{2008}, p. 20) Some text.
  \end{itemize}
\item
  \texttt{Some\ text\ {[}-@citekey{]}.}

  \begin{itemize}
  \tightlist
  \item
    Some text (\protect\hyperlink{ref-leung2008}{2008})
  \end{itemize}
\item
  \texttt{Some\ text\ {[}@citekey1;\ @citekey2{]}.}

  \begin{itemize}
  \tightlist
  \item
    Some text (Leung et al., \protect\hyperlink{ref-leung2008}{2008}; 黃宣範, \protect\hyperlink{ref-huangxuanfan1993}{1993}).
  \end{itemize}
\item
  Prefix \& Suffix

  \begin{itemize}
  \tightlist
  \item
    \texttt{Text\ {[}see\ @citekey1\ pp.45;\ also,\ @citekey2\ ch.\ 2{]}.}
  \item
    Text (see Leung et al., \protect\hyperlink{ref-leung2008}{2008}, p. 45; also, 黃宣範, \protect\hyperlink{ref-huangxuanfan1993}{1993} ch.~2).
  \end{itemize}
\end{itemize}

\hypertarget{ref-manager}{%
\subsection{書目管理軟體}\label{ref-manager}}

這裡建議使用 Zotero 加上 \href{https://retorque.re/zotero-better-bibtex/}{Better BibTeX} 擴充功能。\texttt{citr} 對 Zotero 有額外的支持,且 \textbf{Zotero 能夠控制 citation key 的格式}(例如,last name + year),但其它書目管理軟體如 Endnote 產生的 citation key 對難以讀懂且無法更改其格式。

\hypertarget{multi-lang-cite}{%
\subsection{多語言文獻引用}\label{multi-lang-cite}}

透過 csl 排版引用格式,只能支援單一語言。例如,若將英文格式套用到中文文獻,中文文獻就會出現英文的半形逗點和句點。

\hypertarget{appendix-ux9644ux9304}{%
\appendix \addcontentsline{toc}{chapter}{\appendixname}}


\hypertarget{latex-cite-pkg}{%
\chapter{LaTeX 文獻引用}\label{latex-cite-pkg}}

R Markdown 的 PDF 輸出是透過 Pandoc 的 LaTeX 模板,因此理論上 LaTeX 可以做到的事,也可以透過 R Markdown 達成。目前的問題是

\begin{quote}
LaTeX 本身並未有支援繁體中文格式的文獻引用套件
\end{quote}

經過一段時間的搜尋,發現 biblatex 套件似乎可以定義不同的引用格式\footnote{\url{https://tex.stackexchange.com/questions/417762/different-styles-between-citations-and-bibliography}~\\
  \url{https://tex.stackexchange.com/questions/377308/different-citation-styles-for-the-same-bibliography}},因此,或許可以透過定義新的標點符號,例如將原本引用格式中的\texttt{,}定義成\texttt{,}、\texttt{.}定義成\texttt{。},再透過 \texttt{.bib} 檔中的 \texttt{langid} field 辨識要使用何種引用格式。然而,由於作者本人對 LaTeX 並不熟悉,因此需要這方面高手的協助。

\renewcommand{\href}{\oldhref}

\hypertarget{references}{%
\chapter*{參考資料}\label{references}}
\addcontentsline{toc}{chapter}{參考資料}

\hypertarget{refs}{}
\leavevmode\hypertarget{ref-leung2008}{}%
Leung, A. K.-y., Maddux, W. W., Galinsky, A. D., \& Chiu, C.-y. (2008). Multicultural experience enhances creativity: The when and how. \emph{American Psychologist}, \emph{63}(3), 169--181. \url{https://doi.org/10.1037/0003-066X.63.3.169}

\leavevmode\hypertarget{ref-huangxuanfan1993}{}%
黃宣範. (1993). \emph{語言、社會與族群意識: 臺灣語言社會學的研究}. 臺北市: 文鶴. Retrieved from \url{http://tulips.ntu.edu.tw:1081/record=b1285025*cht}







\end{document}
